%!TeX Options=--shell-escape
\documentclass[bwprint, withouttitlepage]{mathexpthesis}

\usepackage{minted}
\usepackage{hologo}
\usepackage{tikz}
\usetikzlibrary{intersections, calc, bending, decorations.markings, arrows, shapes, positioning, decorations.pathreplacing, shadows, arrows.meta}
\usepackage{url}

\title{数学实验测试文档}
\groupno{1}%学校名称
\membera{A} %队员A
\memberb{B} %队员B
\memberc{C} %队员C
\memberd{D} %队员D

\begin{document}
\maketitle
\tableofcontents
\newpage
\section{模板说明}
\subsection{引入模板}
使用如下\LaTeX 代码:
\begin{minted}[breaklines, breakanywhere]{latex}
    \documentclass[options]{mathexpthesis}
\end{minted}

拥有几个可选项:
\begin{itemize}[itemindent=2em]
    \item \textbf{bwprint/colorprint} 控制打印时的颜色与超链接的边框.
    \item \textbf{withoutpreface} 控制扉页的输出.
    \item \textbf{withouttitlepage} 控制标题页的输出,最好将摘要放在此页.
    \item \textbf{openbookmarknumber} 开启PDF书签的编号,注意此方法会导致编译警告.
\end{itemize}

\subsection{模板参数}
\begin{itemize}[itemindent=2em]
    \item \textbf{$\backslash$title} 实验文档标题,在document中需要使用 \textbf{$\backslash$maketitle},该参数不必须填写.
    \item \textbf{$\backslash$groupno} 小组编号,该参数必须填写.
    \item \textbf{$\backslash$membera,b,c,d} 成员姓名(仅支持恰好四人的小组),该参数必须填写.
\end{itemize}

\subsection{编译}
目前仅支持\hologo{XeLaTeX} 编译.

\section{简单的使用说明}
\subsection{摘要与关键字}
使用方式:
\begin{minted}[breaklines, breakanywhere]{latex}
    \begin{abstract}
        Abstract here.
        \keywords{\LaTeX, typesetting}
    \end{abstract}
\end{minted}

\subsection{节标题}
模板中的指令经过了重定义,但使用方法不变:
\begin{minted}[breaklines, breakanywhere]{latex}
    \section{A section}
    \subsection{A subsection}
    \subsubsection{A subsubsection}
\end{minted}

具体上,修改了一定的间距,重写了一个计数器指令.

\subsection{公式与定理环境}
使用了amsthm, amsmath, amssymb, amsfonts宏包. 使用方式不变.

行内公式使用\$\$即可.行间公式使用equation, equarray, align等方式.

请注意英文的拼写:
\begin{itemize}[itemindent=2em]
    \item lemma 引理
    \item theorem 定理
    \item proof 证明
    \item assumption 假设
    \item definition 定义
    \item example 例
\end{itemize}

注意,此处的计数器是\textbf{章节全局性}的计数器.

还有一些其它的定义,不过用处不算太大,此处不介绍.

这里使用一个我曾经写过的例子.

\begin{theorem}[Risez定理]
    若$f_n \stackrel{m}{\longrightarrow} f$,则$\exists \{f_{n_k}\} \subset \{f_n\}, s.t. \quad f_{n_k} \to f_n \quad a.e.$
\end{theorem}

\begin{proof}
    $f_n \stackrel{m}{\longrightarrow} f$,则$\forall \varepsilon > 0, \delta > 0, \exists n \geqslant 1, s.t. \quad n \geqslant N, m(E(|f_n - f| \geqslant \varepsilon)) < \delta$.
    于是$\forall k \in \mathbb{N}^*$,令$\varepsilon = \frac{1}{k}, \delta = \frac{1}{2^k}$,可以依次选取自然数$n_1, n_2, \cdots, n_k, \cdots, s.t.$
    \begin{equation*}
        m(E(|f_{n_k} - f| \geqslant \frac{1}{k})) < \frac{1}{2^k}.
    \end{equation*}
    下面证明$f_{n_k} \to f \quad a.e.$ \\
    \textbf{令\[E_0 = \bigcap_{N = 1}^{\infty}{\bigcup_{k = N}^{\infty}{E(|f_k - f| \geqslant \frac{1}{k})}}.\]}
    对每个$N = 1, 2, \cdots, $有
    \begin{equation*}
        m(E_0) \leqslant m(\bigcup_{k = N}^{\infty}{E(|f_{n_k} - f| \leqslant \frac{1}{k})}) \leqslant \sum_{k = N}^{\infty}{m(E(|f_{n_k} - f| \geqslant \frac{1}{k}))} < \sum_{k = N}^{\infty}{\frac{1}{2^k}} = \frac{1}{2^{N - 1}}.
    \end{equation*}
    令$N \to \infty$,可知$m(E_0) = 0$.
    由De Morgan公式得,
    \begin{equation*}
        E - E_0 = \bigcup_{N = 1}^{\infty}{\bigcap_{k = N}^{\infty}{E(|f_k - f| \geqslant \frac{1}{k})}}.
    \end{equation*}
    所以$\forall x \in E - E_0, \exists N \geqslant 1, s.t. \quad k \geqslant N$,
    \begin{equation*}
        |f_{n_k}(x) - f(x)| < \frac{1}{k}.
    \end{equation*}
    故$f_{n_k}(x) \to f(x)$,则在$E$上$f_{n_k} \to f \quad a.e.$
\end{proof}

\subsection{图片插入}
对于未指定后缀的图片,将以\textbf{.pdf,.eps,.jpg,.png}的顺序查找.

图片支持的目录:
\begin{minted}[breaklines, breakanywhere]{latex}
    {./}, {figures/}, {figure/}, {pictures/}, {picture/}, {pic/}, {pics/}, {image/}, {images/}
\end{minted}

插入图片的方式:
\begin{minted}[breaklines, breakanywhere]{latex}
    \begin{figure}
        \centering
        \caption{The caption} % caption above
        \includegraphics{figures/...}
        \label{fig-x}
    \end{figure}
\end{minted}

在figure环境中也可以使用TikZ作图. 以下是一个例子.
\begin{figure}
    \centering
    \begin{tikzpicture}[decoration={markings,mark=at position 0.52 with {\arrow{stealth}}}]
        \coordinate (ae) at (90:3cm);
        \coordinate (aun) at (210:3cm);
        \coordinate (am) at (-30:3cm);
        \node[above] at (ae) {几乎处处收敛};
        \node[below left] at (aun) {几乎一致收敛};
        \node[below right] at (am) {依测度收敛};
        \fill[black, opacity=0.8] (ae) circle (2pt);
        \fill[black, opacity=0.8] (am) circle (2pt);
        \fill[black, opacity=0.8] (aun) circle (2pt);
        \draw [postaction={decorate}, thick, black] (ae.south) to [bend right=15] (am.north east);
        \draw [postaction={decorate}, thick, black] (am.north east) to [bend right=15] (ae.south);
        \draw [postaction={decorate}, thick, black] (aun.east) -- (am.west);
        \draw [postaction={decorate}, thick, black] (ae.north east) to [bend right=15] (aun.south);
        \draw [postaction={decorate}, thick, black] (aun.north east) to [bend right=15] (ae.south);
        \node [left, align=justify, font=\small] at (150:2.3cm) {$m(E) < \infty$ \\ Egoroff thm.};
        \node [right, align=justify, font=\small] at (30:2.3cm) {$\exists \{f_{n_k}\} \subset \{f_n\}$ \\ Risez thm.};
        \node [below, align=justify, font=\small] at (-25:1.0cm) {$m(E) < \infty$};
    \end{tikzpicture}
    \caption{Example of TikZ}
    \label{tikz-label}
\end{figure}

\subsection{表格插入}
支持简单的表格,推荐使用Excel2LaTeX工具一键转换各种表格. 一般表格的创建是用tabular, table这样的命令来完成的. 

建议采用三线表. 以下是一个三线表的示例.
\begin{table}[htbp]
  \centering
    \begin{tabular}{lcc}
    \toprule
    \multicolumn{1}{c}{Type} & Biomass($g \cdot m^{-2}$) & Area Required($m^2$) \\
    \midrule
    rain forest & 45000 & 6666.67 \\
    seasonal forest & 35000 & 8571.43 \\
    grassland & 4000  & 75000 \\
    \bottomrule
    \end{tabular}
  \caption{Minimum area needed in different environment}
  \label{tab:addlabel}
\end{table}

\LaTeX 代码由插件生成.
\begin{minted}[breaklines, breakanywhere]{latex}
    \begin{table}[htbp]
      \centering
        \begin{tabular}{lcc}
        \toprule
        \multicolumn{1}{c}{Type} & Biomass($g \cdot m^{-2}$) & Area Required($m^2$) \\
        \midrule
        rain forest & 45000 & 6666.67 \\
        seasonal forest & 35000 & 8571.43 \\
        grassland & 4000  & 75000 \\
        \bottomrule
        \end{tabular}
      \caption{Minimum area needed in different environment}
      \label{tab:addlabel}
    \end{table}
\end{minted}

\subsection{参考文献}
使用以下代码:
\begin{minted}[breaklines, breakanywhere]{latex}
    \begin{thebibliography}{99}
        \bibitem{bib:one} ....
        \bibitem{bib:two} ....
    \end{thebibliography}
\end{minted}

使用$\backslash$cite命令可以实现一个“比较大”的引用\cite{bib:one}.使用新写的$\backslash$upcite命令则是上标引用\upcite{bib:two}.

\subsection{附录}
使用:
\begin{minted}[breaklines, breakanywhere]{latex}
    \begin{appendices}
        \section{...}
        \subsection{...}
    \end{appendices}
\end{minted}

即可创建附录.

附录中可以放置一些实验时产生的很占地方的图片,也可以放置代码.推荐minted宏包,需要Python安装一些额外文件并将其放入系统的环境变量中:
\begin{minted}[breaklines, breakanywhere]{shell}
    pip install Pygments
\end{minted}

从代码生成的角度上,minted的效果优于lstlistings.

\textbf{注意:目前附录仅支持最多两层,否则目录与编号会出错.}

\section{其它注意事项}
\subsection{符号的使用}
句号请写成".",不要使用“。”.

不需要滥用符号,公式标号与定理环境.

所有符号应使用相应的说明,也需要一个整体的表格概述符号的意义.

\subsection{图片与表格的位置}
不要忘加htb.

部分情况不要忘加$\backslash$\textbf{centering}.

\subsection{使用许可与鸣谢}
模板协议为LPPL(直接用就行了).

此文档中除去示例代码的内容,协议为LPPL(即不要未经许可使用我的matlab代码和tikz代码).

由于模板集成了\href{https://github.com/latexstudio/GMCMthesis}{latexstudio/GMCMthesis\upcite{bib:author-1}}与\href{https://github.com/latexstudio/CUMCMthesis}{latexstudio/CUMCMthesis\upcite{bib:author-2}}的文档模板,该部分的版权为版权所属人所有,此模板属于Derived Work(这是鸣谢).

\subsection{已知问题}
Package hyperref Warning: Token not allowed in a PDF string (Unicode). 这个问题不会影响编译,原因是hyperref处理编号时使用了\LaTeX 语法,这些语法不能被PDF接受.最后结果中的书签仍然会正常生成. 目前这个问题通过添加一个额外选项的方法解决.

\subsection{其它}
以下是示例的参考文献内容和示例的附录.如需使用网址,url宏包可能会更好看一点.有时url可能会导致强制换行的问题,此时可以人工换行.

\begin{thebibliography}{99}
    \bibitem{bib:one} upcite reference example
    \bibitem{bib:two} cite reference example
    \bibitem{bib:author-1} \url{https://github.com/latexstudio/GMCMthesis}
    \bibitem{bib:author-2} \url{https://github.com/latexstudio/CUMCMthesis}
\end{thebibliography}

\begin{appendices}
\section{更新记录}
2019/2/25 初次更新

2019/2/26 修复了hyperref链接错误的问题

2019/2/26 调整了一部分代码的顺序,提高可读性

2019/2/27 修改了一个页面编号的bug

\section{一些matlab测试代码}
\subsection{单纯形法-Core}
\begin{minted}[breaklines, breakanywhere]{matlab}
function [xval, fval] = SimplexMethod(c, A, b, epsilon)
    % Simplex method for linear programming.
    % This program is to solve functions like:
    % min f = cx such that Ax = b, x >= 0, b >= 0
    % c, A, b: equation above
    % epsilon: error value when calculating a valid point
    
    if (nargin == 3)
        epsilon = 1e-6;
    end

    if (size(c, 2) ~= size(A, 2)) || (size(c, 1) ~= 1) || ...
       (size(A, 1) ~= size(b, 1)) || (size(b, 2) ~= 1)
        error("Invalid matrix.");
    end

    % Find a solution first.
    extA = [A, eye(size(A, 1))];
    extB = b;
    extC = [zeros(size(c)), -ones(size(b')), 0];
    inX = size(c, 2) + 1: size(c, 2) + size(A, 1);
    extTable = [extA extB; extC]; % Generate a table
    % Manipulate the table
    for i = 1:size(A, 1)
        extTable(end, :) = extTable(end, :) + extTable(i, :);
    end
    % Pivot, step by step
    [val, cind] = max(extTable(end, 1: end - 1));
    while (val > 0)
        minval = inf;
        for i = 1:size(extTable, 1) - 1
            if (extTable(i, cind) * inf >= 0)
                val = extTable(i, end) / extTable(i, cind);
                if (val < minval)
                    minval = val;
                    rind = i; % find row index
                end
            end
        end
        if (isinf(minval))
            error("Unable to find a valid initial point.");
        end
        point = extTable(rind, cind); % find the point
        inX(rind) = cind; % calculate inner columns at the same time
        extTable(rind, :) = extTable(rind, :) ./ point;
        % Elimination
        for i = 1:size(extTable, 1)
            if (i ~= rind)
                extTable(i, :) = extTable(i, :) - extTable(i, cind) .* extTable(rind, :); 
            end
        end
        [val, cind] = max(extTable(end, 1: end - 1));
    end
    if (abs(extTable(end, end)) >= epsilon)
        error("Unable to find a valid initial point.");
    else
        tempX = zeros(size(extTable, 2), 1);
        tempX(inX) = extTable(1: end - 1, end);
        initX = tempX(1: size(c, 2)); % get initial point of original LP
    end
    % change extTable to let useful vectors in
    for i = 1: size(A, 1)
        if (inX(i) > size(c, 2)) % move condition
            ckey = find(extTable(i, 1: size(c, 2)) ~= 0, 1);
            if (~isempty(ckey)) % get the key, apply the change
                point = extTable(i, ckey);
                inX(i) = ckey;
                extTable(i, :) = extTable(i, :) ./ point;
                % Elimination without the bottom
                for j = 1:size(extTable, 1) - 1
                    if (j ~= i)
                        extTable(j, :) = extTable(j, :) - extTable(j, ckey) .* extTable(i, :); 
                    end
                end
            else % don't get the key, remove the line
                warning("Surplus condition found. Removing it.")
                inX(i) = -1; % Tag the key
            end
        end
    end
    table = extTable(inX > 0, [1: size(A, 2), end]); % construct table
    % re-calculate conditional number
    table = [table; zeros(1, size(table, 2))]; % make room first
    table(end, end) = c * initX;
    inX = inX(inX > 0);
    w = c(inX) / table(1: end - 1, inX); % get factor w
    for j = 1: size(table, 2) - 1
        table(end, j) = w * table(1: end - 1, j) - c(j);
    end
    % Pivot, step by step
    [val, cind] = max(table(end, 1: end - 1));
    while (val > 0)
        minval = inf;
        for i = 1:size(table, 1) - 1
            if (table(i, cind) * inf >= 0)
                val = table(i, end) / table(i, cind);
                if (val < minval)
                    minval = val;
                    rind = i; % find row index
                end
            end
        end
        if (isinf(minval))
            error("This problem is unbounded.")
        end
        point = table(rind, cind); % find the point
        inX(rind) = cind; % calculate inner columns at the same time
        table(rind, :) = table(rind, :) ./ point;
        % Elimination
        for i = 1:size(table, 1)
            if (i ~= rind)
                table(i, :) = table(i, :) - table(i, cind) .* table(rind, :); 
            end
        end
        [val, cind] = max(table(end, 1: end - 1));
    end
    tempX = zeros(size(table, 2), 1);
    tempX(inX) = table(1: end - 1, end);
    xval = tempX(1: size(c, 2)); % get initial point of original LP
    fval = table(end, end);
end
\end{minted}
\subsection{单纯形法-Full}
\begin{minted}[breaklines, breakanywhere]{matlab}
function [xval, fval] = SimplexSolver(c, A, b, Aeq, beq, lb, ub, epsilon)
    % Uniform simplex solver for simplex method.
    % Solves problem like:
    % min f = cx, such that A * x <= b, Aeq * x = beq, lb <= x <= ub
    % epsilon: error value when calculating a valid point
    % This program works in larger range than the core simplex method.

    % Autofill
    if (nargin == 3)
        Aeq = []; beq = []; lb = []; ub = []; epsilon = 1e-6;
    elseif (nargin == 5)
        lb = []; ub = []; epsilon = 1e-6;
    elseif (nargin == 6)
        ub = []; epsilon = 1e-6;
    elseif (nargin == 7)
        epsilon = 1e-6;
    end
    origC = c;
    
    % Validate
    if (~isempty(A) && ~isempty(b) && ... 
       ((size(c, 2) ~= size(A, 2)) || (size(c, 1) ~= 1) || ...
       (size(A, 1) ~= size(b, 1)) || (size(b, 2) ~= 1))) || ...
       (~isempty(Aeq) && ~isempty(beq) && ((size(c, 2) ~= size(Aeq, 2)) || ...
       (size(Aeq, 1) ~= size(beq, 1)) || (size(beq, 2) ~= 1))) || ...
       (~isempty(lb) && (size(lb, 2) ~= 1 || size(lb, 1) ~= size(c, 2))) || ...
       (~isempty(ub) && (size(ub, 2) ~= 1 || size(ub, 1) ~= size(c, 2)))
        error("Invalid matrix.");
    end
    
    % Integrate lowerbounds and upperbounds into constraints
    if (isempty(lb))
        lb = -inf(size(c))';
    end
    if (isempty(ub))
        ub = +inf(size(c))';
    end
    
    % create movement table for recovering the data later
    % line 1 demonstrates movement type 
    % value 0 for x - u type + v - x type
    % value 1 for x - u type
    % value 2 for v - x type
    % value 3 for x1 - x2 type
    % line 2 demonstrates movement place
    % value n for place n
    % value -1 for not taking a place
    
    movement_table = zeros(2, size(c, 2));
    place = size(c, 2) + 1;
    for i = 1: size(c, 2)
        if (ub(i) - lb(i) < 0)
            error("lowerbound is greater than upperbound.");
        end       
        if (lb(i) ~= -inf) && (ub(i) ~= +inf)
            movement_table(1, i) = 0;
            movement_table(2, i) = -1;
            % Add constraint for full bounded condition
            if (~isempty(A))
                b(:) = b(:) - A(:, i) * lb(i);
            end
            if (~isempty(Aeq))
                beq(:) = beq(:) - Aeq(:, i) * lb(i);
            end
            temprow = zeros(size(c));
            temprow(i) = 1;
            A = [A; temprow];
            b = [b; ub(i) - lb(i)];
        elseif (lb(i) ~= -inf) && (ub(i) == +inf)
            movement_table(1, i) = 1;
            movement_table(2, i) = -1;
            if (~isempty(A))
                b(:) = b(:) - A(:, i) * lb(i);
            end
            if (~isempty(Aeq))
                beq(:) = beq(:) - Aeq(:, i) * lb(i);
            end 
        elseif (lb(i) == -inf) && (ub(i) ~= +inf)
            movement_table(1, i) = 2;
            movement_table(2, i) = -1;
            c(i) = -c(i);
            if (~isempty(A))
                b(:) = b(:) - A(:, i) * ub(i);
                A(:, i) = -A(:, i);
            end
            if (~isempty(Aeq))
                beq(:) = beq(:) - Aeq(:, i) * ub(i);
                Aeq(:, i) = -Aeq(:, i);
            end
        elseif (lb(i) == -inf) && (ub(i) == +inf)
            movement_table(1, i) = 3;
            movement_table(2, i) = place;
            place = place + 1;
            % Add variable for unbounded condition
            c = [c, -c(i)];
            if (~isempty(A))
                A = [A, -A(:, i)];
            end
            if (~isempty(Aeq))
                Aeq = [Aeq, -Aeq(:, i)];
            end
        end
    end
    
    % Add surplus variables for inequalities and make it stardardized
    for i = 1: size(A, 1)
        tempcol = zeros(size(A, 1), 1);
        if b(i) < 0
            A(i, :) = -A(i, :);
            b(i) = -b(i);
            tempcol(i) = -1;
            A = [A, tempcol];
            c = [c, 0];
        else
            tempcol(i) = 1;
            A = [A, tempcol];
            c = [c, 0];
        end
        if (~isempty(Aeq))
            Aeq = [Aeq, zeros(size(Aeq, 1), 1)];
        end
    end
    
    % Make equality constraints startardized
    for i = 1: size(Aeq, 1)
        if (beq(i) < 0)
            Aeq(i, :) = -Aeq(i, :);
            beq(i) = -beq(i);
        end
    end
    
    % Solve the LP using the core simplex method module
    [tempxval, ~] = SimplexMethod(c, [A; Aeq], [b; beq], epsilon);

    xval = nan(size(origC, 2), 1);
    % From movement_table to recover final value
    for i = 1: size(movement_table, 2)
        switch (movement_table(1, i))
            case {0, 1}
                xval(i) = tempxval(i) + lb(i);
            case 2
                xval(i) = ub(i) - tempxval(i);
            case 3
                xval(i) = tempxval(i) - tempxval(movement_table(2, i));
        end
    end
    fval = origC * xval;
    
end
    
\end{minted}
\end{appendices}

\end{document}